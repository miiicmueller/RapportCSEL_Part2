% classe du document
\documentclass[11pt,a4paper]{report}

% packages
\usepackage[T1]{fontenc}
\usepackage{mathtools}
\usepackage[utf8]{inputenc}
\usepackage[francais]{babel} 
\usepackage{lipsum}
\usepackage{footnote}
\usepackage{cases}
\selectlanguage{francais}
\usepackage{lastpage}

% insertion d'image
\usepackage{graphicx}
\usepackage{float} % pour le placement des images
\usepackage{subfig} % pour mettre des figures cote-a-cote
\usepackage{epsf} % pour separer une page en deux (minipage)
\usepackage{multicol} % pour creer des colonnes (multicol)
\graphicspath{	{Ressources/images/}
			{Ressources/Logos/}
			{Ressources/Graphiques/}
			{Ressources/Graphiques/Processus/}
			{Ressources/Graphiques/SA13/}} % Chemin predefinit pour classer les images
\DeclareGraphicsExtensions{.pdf,.png,.jpg} % declaration des extensions pour ne pas les ecrire tout le temps

% style de mise en page
\usepackage{geometry}
\geometry{top=2.5cm, bottom=2.5cm,left=2.3cm, right=2.3cm}
\headheight = 20pt
\usepackage{fancyhdr}
\pagestyle{fancy}
\lhead{\entetegauche}
\chead{}
\rhead{\entetedroite}
\lfoot{}
\cfoot{}
\rfoot{\thepage}

% creation graphique
\usepackage[usenames, dvipsnames]{xcolor}
\usepackage{tikz}
\usetikzlibrary{shapes}

% listings
\usepackage{verbatim}
\usepackage{listings}
%\usepackage{xcolor}
\definecolor{grey}{rgb}{0.96,0.96,0.96}
\definecolor{dkgreen}{rgb}{0,0.6,0}
\definecolor{mauve}{rgb}{0.58,0,0.82}
\definecolor{red}{rgb}{1,0,0}

\lstset{
backgroundcolor=\color{grey}, 
breaklines=true, 
framexleftmargin=1mm, 
xleftmargin=5mm,
xrightmargin=1mm,
showstringspaces=false,
basicstyle=\scriptsize,%\small,
identifierstyle=\normalfont,
% insertion label : //(*\label{comment}*)
}

\lstset{literate=
	{æ}{{\ae}}1
	{å}{{\aa}}1
	{ø}{{\o}}1
	{Æ}{{\AE}}1
	{Å}{{\AA}}1
	{Ø}{{\O}}1
	{é}{{\'e}}1
	{è}{{\`e}}1
	{à}{{\`a}}1
	{É}{{\'E}}1
	{È}{{\`E}}1
	{À}{{\`A}}1
	{Ä}{{\¨a}}1
	{©}{{\copyright}}1
	{‐}{{-}}1
	{ù}{{\`u}}1
	{ô}{{\^o}}1
}

\lstdefinestyle{Console}{
	language=c,
	showspaces=false,
	backgroundcolor=	\color{black},
	basicstyle=		\color{green}\scriptsize, 
	keywordstyle=		\color{green}, 
	commentstyle=	\color{green}, 
	stringstyle=		\color{white},
	title=\lstname,
	numbers = left, numberstyle=\tiny\color{black}
	}

\lstdefinestyle{C}{
	language=c, showspaces=false, 
	keywordstyle=		\color{blue}\scriptsize, 
	commentstyle=	\color{dkgreen}\scriptsize\sffamily,  
	stringstyle=		\color{mauve}\scriptsize, 
	title=\lstname,
	escapeinside={//(*}{*)},
	morekeywords={},
	numbers = left, numberstyle=\tiny\color{black}}

\lstdefinestyle{Matlab}{
	language=Matlab, showspaces=false, 
	keywordstyle=		\color{blue}\scriptsize, 
	commentstyle=	\color{dkgreen}\scriptsize\sffamily,  
	stringstyle=		\color{mauve}\scriptsize, 
	title=\lstname,
	escapeinside={(*}{*)},
	morekeywords={},
	numbers = left, numberstyle=\tiny\color{black}}
	
% commandes custom
\providecommand{\abs}[1]{\lvert#1\rvert} % valeur absolue
\providecommand{\norm}[1]{\lVert#1\rVert} % norme
\providecommand{\der}[2]{\frac{\mathrm{d}#1}{\mathrm{d}#2}} % derivee
\newcommand{\tmp}[1]{{\color{red}\textbf{\\#1\\} \marginpar{\color{red}\Huge{\Huge{*}}}}} % note temoraire dans la marge
\newcommand{\tmpnew}[1]{{\color{red}\Huge{*}}\marginpar{\color{red}#1}} % note temoraire dans la marge new
\newcommand{\rem}[1]{\textit{\\ Remarque : {#1}\\}} % note temporaire dans le texte


% Couleur disponibles avec xcolor
%  – De jaune à orange : GreenYellow, Yellow, Goldenrod, Dandelion, Apricot, Peach, Melon, YellowOrange,
%  – De orange à rouge : Orange, BurntOrange, Bittersweet, RedOrange, Mahogany, Maroon, BrickRed, Red,
%  – De rouge à rose : OrangeRed, RubineRed, WildStrawberry, Salmon, CarnationPink, Magenta, VioletRed, Rhodamine,
%  – De rose à violet : Mulberry, RedViolet, Fuchsia, Lavender, Thistle, Orchid, DarkOrchid, Purple,
%  – De violet à bleu : Plum, Violet, RoyalPurple, BlueViolet, Periwinkle, CadetBlue, CornflowerBlue, MidnightBlue,
%  – De bleu à bleu clair : NavyBlue, RoyalBlue, Blue, Cerulean, Cyan, ProcessBlue, SkyBlue, Turquoise
%  – De bleu clair à vert : TealBlue, Aquamarine, BlueGreen, Emerald, JungleGreen, SeaGreen, Green, ForestGreen,
%  – De vert à brun : PineGreen, LimeGreen, YellowGreen, SpringGreen, OliveGreen, RawSienna, Sepia, Brown, Tan






